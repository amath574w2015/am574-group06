\documentclass[10pt,a4paper]{article}
\usepackage[utf8]{inputenc}
\usepackage{amsmath}
\usepackage{amsfonts}
\usepackage{amssymb}
\begin{document}
\noindent Kelsey Maass \\
Brisa Davis \\
AMATH 574\\
\vspace{0.2in}

We are planning on studying different second order traffic flow models. Currently, we plan to study a second order models that failed to accurately capture the behavior seen in reality. We will examine the Payne and Whitham (PW) model, and explore the logical flaws in the arguments that have been used to derive this higher order model \cite{Daganzo1995}. With this study we will illustrate the importance of validation in modeling, and highlight the fact that simply having a higher order code does not equate to more accurately modeling a given physical system.

We also examine the more accurate Aw Rascle model presented in \cite{AwRascle2000}, which uses a convective derivative to resolve inconsistencies present in older models. The main critique this newer model has of the PW model is that the PW model uses fluid flow behavior to generate their model, while car behavior is in reality not accurately captured by fluid flow equations. As an example, fluid particles receive and respond to stimuli from both the front and the back whereas a car mostly responds to frontal stimuli. 

We will compare both models to show that the PW model gives nonphysical results while Aw Rascle model more accurately reflects reality. 

\bibliography{ProjectProposal}{}
\bibliographystyle{plain}
\end{document}